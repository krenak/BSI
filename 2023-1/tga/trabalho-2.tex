% 1o trabalho de TGA
% Anderson Fraga - Marco 2023

\documentclass[11pt]{article}
\usepackage[a4paper,includeheadfoot,margin=2.54cm]{geometry}
% \usepackage{cmbright}
\usepackage[OT1]{fontenc}
% \usepackage{antiqua}
% \usepackage[T1]{fontenc}
% \usepackage{notomath}
\usepackage{parskip}

\begin{document}

\underline{\textbf{Segundo trabalho de Teoria Geral de Administração - Parte 2}}\par
\textbf{Sistemas de Informação}\\
\textbf{Instituto Federal do Espírito Santo}\\
Campus Serra - Espírito Santo\par
\textbf{Teoria Geral da Administração}\\
Prof. Ronaldo Marques\par
Anderson A. Fraga (20222BSI0482)\\
\texttt{aafrg@tuta.io}\\  %\texttt formats the text to a typewriter style font


\noindent \textbf{Questão 1) Como podemos definir a Teoria Neoclássica (TN)?}\par
É uma abordagem econômica que surgiu como uma evolução da Teoria Clássica no final do século XIX e início do século XX. A TN busca combinar os princípios da economia clássica com conceitos e ideias adicionais, incorporando avanços teóricos e empíricos posteriores. Ela foca na importância do mercado e na alocação eficiente de recursos, reconhecendo a existência de uma economia de mercado com base em livre concorrência, onde a oferta e a demanda interagem para determinar preços e quantidades de bens e serviços.

\noindent \textbf{Questão 2) Qual é a visão do ser humano na TN?}\par
O ser humano é visto como um agente econômico racional que busca tomar decisões que maximizem sua utilidade ou seu lucro. Essa abordagem parte do pressuposto de que os indivíduos têm preferências claras e consistentes e que são capazes de tomar decisões informadas com base nessas preferências.

\noindent \textbf{Questão 3) De forma resumida, liste as principais características da TN}\par
\begin{enumerate}
    \item \textbf{Ênfase no mercado}: A TN enfatiza a importância do mercado como um mecanismo de alocação de recursos, onde a interação entre oferta e demanda determina preços e quantidades de bens e serviços;
    \item \textbf{Racionalidade individual}: Os indivíduos são vistos como agentes econômicos racionais, que tomam decisões informadas e buscam maximizar sua utilidade ou lucro;
    \item \textbf{Utilidade marginal}: A TN incorpora o conceito de utilidade marginal, que considera que o valor de um bem ou serviço para um indivíduo diminui à medida que ele consome mais unidades desse bem ou serviço.
    \item \textbf{Equilíbrio geral}: A TN enfatiza o equilíbrio geral da economia, onde a interação de oferta e demanda leva a um ponto de equilíbrio em que os preços e quantidades são determinados.
\end{enumerate}
%\pagebreak
\noindent \textbf{Questão 4) Quais os elementos centrais do chamado processo administrativo “PODC”?}
\par O processo administrativo "PODC" é uma abordagem que visa aprimorar a gestão e o desempenho organizacional. O acrônimo "PODC" refere-se aos elementos centrais desse processo: Planejamento, Organização, Direção e Controle.
\begin{enumerate}
    \item \emph{Planejamento}: é a fase inicial do processo administrativo em que são estabelecidos os objetivos, metas e estratégias da organização. Envolve a definição de planos de curto e longo prazo, identificação de recursos necessários, análise de riscos e incertezas, e a tomada de decisões sobre as ações a serem tomadas; e
    \item \emph{Organização}: trata da estruturação e design da organização para alcançar os objetivos estabelecidos no planejamento. Envolve a divisão de tarefas, alocação de recursos, estabelecimento de linhas de autoridade e responsabilidade, criação de departamentos e unidades funcionais, e definição de relações hierárquicas e de comunicação;
    \item \emph{Direção}: refere-se à liderança e supervisão das atividades da organização. Nessa fase, os gestores coordenam e orientam os colaboradores para executar as tarefas definidas no planejamento. Inclui a motivação da equipe, comunicação efetiva, treinamento, delegação de responsabilidades e resolução de conflitos;
    \item \emph{Controle}: é a etapa em que se monitora e avalia o desempenho organizacional em relação aos planos estabelecidos. Envolve a definição de indicadores de desempenho, coleta e análise de dados, comparação dos resultados com as metas estabelecidas, identificação de desvios e tomada de medidas corretivas, se necessário. O controle permite que a organização identifique problemas, faça ajustes e melhore continuamente seu desempenho.
\end{enumerate}
% \pagebreak
\noindent \textbf{Questão 5) Explique a diferença entre Eficiência e Eficácia.}\par
\begin{itemize}
    \item \textbf{\emph{Eficiência}}: Eficiência: A eficiência refere-se à capacidade de realizar uma tarefa, atividade ou processo utilizando os recursos disponíveis de maneira ótima. É a relação entre os recursos utilizados e os resultados alcançados. Uma organização eficiente busca minimizar o desperdício, reduzir custos e otimizar o uso dos recursos, como tempo, dinheiro, mão de obra e materiais.
    \item \textbf{\emph{Eficácia}}: refere-se à capacidade de atingir os resultados desejados, alcançar metas e objetivos estabelecidos. Está relacionada à obtenção de resultados satisfatórios e à realização dos propósitos da organização.
\end{itemize}

\noindent \textbf{Questão 6) Explique a Teoria Estruturalista de Amitai Etzioni. Qual era a sua visão?}\par
A Teoria Estruturalista de Amitai Etzioni é uma abordagem da sociologia organizacional que enfatiza a importância das estruturas organizacionais e das relações de poder na compreensão do comportamento humano nas organizações.\par
A visão de Etzioni na Teoria Estruturalista pode ser resumida em \textbf{três pontos-chave}:
\begin{enumerate}
    \item A \emph{Análise das estruturas organizacionais}: Etzioni argumentou que as organizações podem ser entendidas por meio da análise das suas estruturas formais e informais. A estrutura formal se refere aos elementos visíveis da organização, como hierarquia, divisão de trabalho, departamentos e regras. A estrutura informal, por outro lado, diz respeito aos aspectos não oficiais, como redes sociais, normas não escritas e relações pessoais;
    \item \emph{Relações de poder e controle}: Etzioni argumentou que as organizações são caracterizadas por relações de poder e controle. Ele identificou três tipos de poder nas organizações: coercitivo (baseado na aplicação de punições ou sanções), utilitário (baseado em recompensas ou incentivos) e normativo (baseado em normas sociais e valores compartilhados).;
    \item \emph{Comportamento organizacional}: Etzioni acreditava que o comportamento dos indivíduos nas organizações é influenciado por incentivos, sanções e normas sociais. Ele argumentou que os indivíduos são motivados por uma combinação de recompensas materiais, como salário e benefícios, e recompensas sociais, como prestígio e reconhecimento. Além disso, Etzioni enfatizou que as normas sociais desempenham um papel fundamental na conformidade e no comportamento ético dentro das organizações.
\end{enumerate}

\noindent \textbf{Questão 7) Explique a Teoria da Burocracia de acordo com Max Weber.}\par
É uma abordagem sociológica que busca compreender e explicar a forma como as organizações funcionam. Weber desenvolveu essa teoria no início do século XX, destacando características fundamentais da burocracia como uma forma de organização eficiente e racional. Segundo Weber, a burocracia é um tipo ideal, ou seja, um modelo teórico que descreve as características mais importantes desse tipo de organização.
% \pagebreak

\noindent \textbf{Questão 8) Para Weber, o que diferencia poder, dominação e autoridade?}\par
\textbf{Poder} refere-se à capacidade de impor a própria vontade, independentemente do consentimento dos outros; \textbf{dominação} é uma forma específica de exercício do poder que envolve o consentimento voluntário; e \textbf{autoridade} é a forma legítima de dominação reconhecida e aceita pela sociedade ou grupo.\par

\noindent \textbf{Questão 9) Cite três contribuições e três limitações da Burocracia}\par
\textbf{Contribuições da Burocracia:}
\begin{enumerate}
    \item \emph{Eficiência e racionalidade}: A burocracia é projetada para alcançar eficiência por meio da divisão de trabalho e especialização. As tarefas são distribuídas de forma hierárquica, permitindo que os funcionários se concentrem em suas áreas de expertise;
    \item \emph{Imparcialidade e igualdade}: A burocracia estabelece regras e regulamentos formais, aplicáveis a todos os membros da organização. Essa abordagem impessoal promove a igualdade de tratamento e evita a discriminação ou favoritismo com base em características pessoais, garantindo uma base mais justa para as decisões organizacionais;
    \item \emph{Previsibilidade e estabilidade}: A burocracia se baseia em procedimentos padronizados e previsíveis. As regras e regulamentos estabelecidos fornecem um quadro claro para as operações organizacionais, permitindo a antecipação e o planejamento adequado das atividades.
\end{enumerate}
\textbf{Limitações da Burocracia:}
\begin{enumerate}
    \item \emph{Rigidez e inflexibilidade}: A burocracia tende a ser caracterizada por uma estrutura hierárquica rígida e procedimentos burocráticos detalhados. Essa rigidez pode tornar a organização lenta em sua capacidade de adaptação a mudanças e inovações;
    \item \emph{Excesso de papelada e procedimentos burocráticos}: A burocracia é frequentemente associada a uma quantidade significativa de procedimentos complexos. Isso pode levar a uma burocratização excessiva, onde os funcionários ficam sobrecarregados com tarefas administrativas e a tomada de decisões pode ser adiada devido a uma cadeia hierárquica longa e processos demorados;
    \item \emph{Despersonalização e alienação}: A ênfase na impessoalidade na burocracia pode levar à despersonalização dos relacionamentos e à alienação dos funcionários. A estrutura hierárquica e a natureza formal das relações podem reduzir a motivação e o senso de identidade dos indivíduos, resultando em um ambiente de trabalho desengajado e desmotivador.
\end{enumerate}

\noindent \textbf{Questão 10) Explique a abordagem da Teoria Contingencial.}\par
A Teoria Contingencial é uma abordagem na administração que enfatiza que não existe uma única forma de organização ou estilo de gestão que seja eficaz em todas as situações. Em vez disso, essa teoria reconhece que as práticas de gestão e as estruturas organizacionais devem ser adaptadas às circunstâncias específicas em que a organização opera. A abordagem da Teoria Contingencial destaca que diferentes fatores externos e internos influenciam o funcionamento e o desempenho das organizações.\par

\noindent \textbf{Questão 11) Explique o que é Administração estratégica e o que ela envolve.}\par
Administração estratégica é uma abordagem de gestão que se concentra na formulação e implementação de estratégias para alcançar os objetivos de longo prazo de uma organização. Ela envolve a análise do ambiente externo e interno da organização, o estabelecimento de metas e objetivos, o desenvolvimento de planos de ação e a alocação de recursos para alcançar esses objetivos de forma eficaz. A administração estratégica envolve uma série de etapas e processos, que podem variar de acordo com o modelo utilizado.\par

\noindent \textbf{Questão 12) Liste as 7 etapas do Planejamento Estratégico.}\par
\begin{enumerate}
    \item \emph{Análise do ambiente externo e interno}: são realizadas análises do ambiente externo e interno da organização. A análise do ambiente externo envolve a identificação de oportunidades e ameaças no mercado, a análise da concorrência e a consideração de fatores econômicos, políticos, sociais e tecnológicos que afetam a organização. A análise do ambiente interno avalia os recursos, as capacidades e as competências da organização, identificando suas forças e fraquezas;
    \item \emph{Definição da visão, missão e valores}: A visão representa a imagem futura que a organização busca alcançar, enquanto a missão descreve o propósito fundamental da organização. Os valores são os princípios e crenças que orientam as ações da organização. Esses elementos fornecem uma base para a definição dos objetivos e estratégias;
    \item \emph{Estabelecimento de objetivos e metas}: Com base na visão, missão e valores, são definidos os objetivos de longo prazo da organização. Os objetivos devem ser específicos, mensuráveis, alcançáveis, relevantes e temporais (SMART), e devem estar alinhados com a visão e a missão da organização;
    \item \emph{Formulação de estratégias}: Nesta etapa, as estratégias são formuladas para atingir os objetivos estabelecidos. Isso envolve a identificação de abordagens e planos de ação que permitam à organização alcançar suas metas. As estratégias podem abordar questões como vantagem competitiva, crescimento, diversificação, inovação;
    \item \emph{Desenvolvimento de planos de ação}: Os planos de ação são elaborados para detalhar as etapas necessárias para implementar as estratégias. Eles definem as tarefas, os prazos, os recursos necessários e as responsabilidades para a execução das ações planejadas. Os planos de ação podem abranger diferentes áreas funcionais da organização;
    \item \emph{Implementação}: Aqui os planos de ação são colocados em prática. A implementação envolve a alocação de recursos, a coordenação das atividades, a comunicação clara das diretrizes e a gestão das mudanças necessárias. É importante envolver as pessoas relevantes da organização e garantir o comprometimento e o alinhamento com as estratégias e objetivos;
    \item \emph{Monitoramento e avaliação}: O processo de Planejamento Estratégico não termina com a implementação. É essencial monitorar o progresso e avaliar os resultados alcançados em relação aos objetivos estabelecidos. Isso envolve a coleta e análise de dados, a comparação com os indicadores de desempenho estabelecidos e a identificação de eventuais ajustes e melhorias necessárias. A aprendizagem contínua e a adaptação são fundamentais nessa etapa.
\end{enumerate}

\noindent \textbf{Questão 13) Liste os 5p’s da estratégica segundo Henry Mintzberg.}\par
\begin{enumerate}
    \item \textbf{Plano (\emph{Plan})}: Refere-se à formulação de um plano estratégico formal e sistemático para atingir os objetivos organizacionais. Isso envolve a definição de metas e objetivos, a identificação de estratégias e a elaboração de um roteiro detalhado para a implementação;
    \item \textbf{Padrão (\emph{Pattern})}: O padrão se refere ao comportamento passado da organização, observando como as estratégias têm sido implementadas ao longo do tempo e quais padrões e tendências podem ser identificados. Os padrões podem fornecer insights sobre as escolhas estratégicas da organização e sua consistência ao longo do tempo;
    \item \textbf{Posição (\emph{Position})}: A posição estratégica envolve a identificação da posição única ocupada pela organização em relação ao mercado e à concorrência. Isso inclui a análise da vantagem competitiva, do posicionamento no mercado, das características distintivas e do valor percebido pelos clientes.
    \item \textbf{Perspectiva (\emph{Perspective})}: A perspectiva estratégica está relacionada à mentalidade, visão e direção estratégica da organização. Envolve a compreensão das crenças, valores, cultura organizacional e identidade que moldam as decisões estratégicas;
    \item \textbf{Plano Estratégico (\emph{Ploy})}: O \emph{ploy} refere-se a manobras estratégicas ou ações específicas que são tomadas para obter uma vantagem competitiva em relação aos concorrentes. Essas manobras podem incluir movimentos de marketing, aquisições estratégicas, mudanças de preço, parcerias estratégicas ou qualquer ação tática que busque obter uma posição de destaque no mercado.
\end{enumerate}

\noindent \textbf{Questão 14) Quais são os três desafios a observar por organizações que pretendem implantar a Administração estratégica (AE)?}
\begin{enumerate}
    \item \textbf{Mudança de mentalidade e cultura organizacional}: A Administração Estratégica envolve uma abordagem de longo prazo e uma mentalidade estratégica que pode exigir uma mudança significativa na cultura e nas práticas organizacionais. Muitas vezes, as organizações estão acostumadas a operar de forma reativa, focando no curto prazo e nas atividades rotineiras. Implementar a AE requer uma mentalidade voltada para o futuro, focada em oportunidades, inovação e adaptação às mudanças. É essencial que a liderança e os membros da organização estejam alinhados e comprometidos com essa nova abordagem;
    \item \textbf{Coleta e análise de informações relevantes}: A AE exige uma análise aprofundada do ambiente externo e interno da organização, a fim de tomar decisões estratégicas fundamentadas. Isso requer a coleta e a análise de informações relevantes, como dados de mercado, tendências, concorrência, tecnologia e recursos organizacionais;
    \item \textbf{Implementação eficaz}: A implementação das estratégias é outro desafio significativo. Muitas vezes, as organizações enfrentam dificuldades em traduzir as estratégias em ações concretas e integrá-las às operações diárias. A implementação eficaz requer o engajamento e a colaboração de todas as partes envolvidas, alocação adequada de recursos, estabelecimento de metas e indicadores de desempenho claros, além de uma comunicação eficaz.
\end{enumerate}

\noindent \textbf{Questão 15) Explique a Teoria de Sistemas de acordo com Bertalanffy.}\par
A Teoria de Sistemas, proposta pelo biólogo austríaco Ludwig von Bertalanffy, é uma abordagem interdisciplinar que busca compreender e descrever os princípios e padrões comuns encontrados em diversos tipos de sistemas complexos. De acordo com Bertalanffy, um sistema é um conjunto de elementos inter-relacionados que funcionam juntos para alcançar um objetivo comum. Os sistemas podem ser encontrados em diversos níveis e áreas, como biologia, ecologia, psicologia, sociologia, administração e outras disciplinas. A teoria de sistemas considera que todos esses sistemas compartilham características e princípios básicos.\par

\noindent \textbf{Questão 16) Explique o que vem a ser Teoria da Gestão da Qualidade.}\par
A Teoria da Gestão da Qualidade é um conjunto de princípios e práticas que tem como objetivo alcançar e manter altos padrões de qualidade em uma organização. Essa teoria é baseada na ideia de que a qualidade é um fator fundamental para o sucesso de uma empresa e deve ser uma preocupação em todos os níveis e processos organizacionais. A gestão da qualidade envolve a aplicação de abordagens, metodologias e ferramentas específicas para identificar, medir, controlar e melhorar a qualidade dos produtos, serviços e processos de uma organização.\par

\noindent \textbf{Questão 17) Como a Gestão da Qualidade promove vantagem competitiva para uma organização?}\par
A Gestão da Qualidade promove vantagem competitiva ao fornecer produtos ou serviços que atendem ou excedem as expectativas dos clientes, melhoraram continuamente a eficiência operacional, constituem uma reputação positiva e desenvolvem uma cultura de qualidade. Esses fatores combinados podem ajudar a organização a se destacar no mercado, atrair e reter clientes e se posicionar como líder em seu setor.

\end{document}