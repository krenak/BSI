% 1o trabalho de TGA
% Anderson Fraga - Marco 2023

\documentclass[11pt]{article}
\usepackage[a4paper,includeheadfoot,margin=2.54cm]{geometry}
% \usepackage{cmbright}
\usepackage[OT1]{fontenc}
% \usepackage{antiqua}
% \usepackage[T1]{fontenc}
% \usepackage{notomath}
\usepackage{parskip}

\begin{document}

\underline{\textbf{Segundo trabalho de Teoria Geral de Administração - Parte 1}}\par
\textbf{Sistemas de Informação}\\
\textbf{Instituto Federal do Espírito Santo}\\
Campus Serra - Espírito Santo\par
\textbf{Teoria Geral da Administração}\\
Prof. Ronaldo Marques\par
Anderson A. Fraga (20222BSI0482)\\
\texttt{aafrg@tuta.io}\\  %\texttt formats the text to a typewriter style font


\noindent \textbf{Questão 1) Qual a diferença entre a ênfase da Administração Sistemática e da Administração Cientifica?}\par
A Administração Sistemática enfatiza a visão sistêmica, a melhoria contínua e a adaptação ao ambiente, enquanto a Administração Científica enfatiza a eficiência operacional, a padronização e a aplicação de métodos científicos para aumentar a produtividade. Ambas as abordagens contribuíram para o desenvolvimento da teoria e prática da administração, cada uma com suas perspectivas e contribuições específicas.\par

\noindent \textbf{Questão 2) Explique como Taylor definiu o método para se obter eficiência do trabalho na Administração Cientifica.}\par
O método proposto por Taylor tinha como objetivo alcançar uma eficiência máxima do trabalho, reduzindo o desperdício, aumentando a produtividade e melhorando a qualidade. Embora tenha sido criticado por uma abordagem mecanicista e desconsiderar aspectos humanos, o trabalho de Taylor foi fundamental para o desenvolvimento da Administração Científica e influenciou profundamente a forma como o trabalho é organizado e gerenciado até hoje.\par

\noindent \textbf{Questão 3) Explique o que vem a ser \emph{Homo Economicus} na Administração Cientifica.}\par
Na Administração Científica, o termo "Homo Economicus" refere-se a um conceito que retrata os trabalhadores como seres racionais e motivados principalmente por interesses econômicos. Esse conceito foi influenciado pelas ideias do economista clássico Adam Smith e foi adotado por Frederick Taylor, um dos principais proponentes da Administração Científica.

De acordo com essa visão, o Homo Economicus é retratado como um trabalhador que busca maximizar sua utilidade e benefícios financeiros. Ele é visto como um indivíduo motivado primordialmente pelo incentivo financeiro e pela busca do seu próprio interesse econômico. Nessa perspectiva, os trabalhadores são vistos como agentes racionais que tomam decisões com base na maximização de seus ganhos pessoais.

\noindent \textbf{Questão 4) Relacione quatro críticas à Administração Cientifica.}\par
\begin{enumerate}
    \item \textbf{Abordagem mecanicista do trabalho}: A Administração Científica é frequentemente criticada por sua abordagem mecanicista do trabalho, que trata os trabalhadores como meros componentes de um sistema de produção. Essa abordagem desconsidera a complexidade e a natureza humana do trabalho, reduzindo-o a uma série de tarefas simplificadas e repetitivas. Isso pode levar à desmotivação, falta de criatividade e falta de satisfação dos trabalhadores.
    \item \textbf{Ênfase excessiva na eficiência em detrimento do bem-estar dos trabalhadores}: A Administração Científica enfatiza fortemente a eficiência e a produtividade, muitas vezes em detrimento do bem-estar e da qualidade de vida dos trabalhadores. Ao priorizar a maximização da produção e dos lucros, os aspectos humanos, como a qualidade de vida, o equilíbrio entre trabalho e vida pessoal e a realização pessoal, podem ser negligenciados. Isso pode resultar em altos níveis de estresse, insatisfação e rotatividade de funcionários.
    \item \textbf{Falta de consideração pelos aspectos sociais e psicológicos do trabalho}: A Administração Científica tende a ignorar os aspectos sociais e psicológicos do trabalho, focando exclusivamente na eficiência e na produtividade. Ela não leva em conta as relações interpessoais, a cooperação, a motivação intrínseca e outros fatores que influenciam o desempenho e a satisfação dos trabalhadores. Essa abordagem limitada pode levar a uma falta de engajamento dos funcionários e a um clima organizacional negativo.
    \item \textbf{Desconsideração da capacidade e conhecimento dos trabalhadores}: A Administração Científica muitas vezes desconsidera a experiência e o conhecimento dos trabalhadores no processo de tomada de decisão. As decisões são centralizadas na gerência, e os trabalhadores têm pouco ou nenhum poder de influência sobre suas tarefas e processos de trabalho. Isso pode resultar em uma subutilização do potencial dos trabalhadores e na perda de oportunidades de melhoria e inovação no trabalho.
\end{enumerate}

\noindent \textbf{Questão 5) Qual era o foco da Administração Clássica preconizada Henri Fayol.}\par
O foco da Administração Clássica preconizada por Henri Fayol era a eficiência e a organização das atividades dentro de uma empresa. Fayol desenvolveu uma abordagem sistemática para a administração, baseada em sua experiência como diretor geral de uma grande empresa industrial no início do século XX. Fayol identificou cinco funções essenciais da administração: planejamento, organização, coordenação, comando e controle.\par

\noindent \textbf{Questão 6) Relacione os 14 princípios para uma administração eficaz segundo Fayol.}\par
Os 14 princípios de administração propostos por Henri Fayol são:
\begin{enumerate}
    \item Divisão do trabalho: Deve haver uma divisão clara e especialização das tarefas para aumentar a eficiência e a produtividade;
    \item Autoridade e responsabilidade: A autoridade deve estar acompanhada da responsabilidade correspondente. Os gerentes têm autoridade para dar ordens, mas também são responsáveis pelo resultado;
    \item Disciplina: Os funcionários devem obedecer às regras e regulamentos estabelecidos pela organização;
    \item Unidade de comando: Cada funcionário deve receber ordens de apenas um superior, evitando conflitos de autoridade;
    \item Unidade de direção: A organização deve ter um plano único e uma única direção para alcançar os objetivos;
    \item Subordinação dos interesses individuais aos interesses gerais: Os interesses individuais dos funcionários devem estar subordinados aos interesses e objetivos da organização;
    \item Remuneração justa: Os salários e benefícios devem ser justos para garantir a satisfação dos funcionários e a eficiência da organização;
    \item Centralização: O grau de centralização das decisões deve ser equilibrado de acordo com as necessidades e capacidades da organização;
    \item Hierarquia: Deve existir uma cadeia de comando clara, com níveis hierárquicos bem definidos;
    \item Ordem: Os recursos e os funcionários devem ser organizados de forma adequada para facilitar a eficiência e a produtividade;
    \item Equidade: Os gerentes devem tratar os funcionários com justiça e imparcialidade;
    \item Estabilidade do pessoal: A rotatividade de pessoal deve ser reduzida para garantir a continuidade e o desenvolvimento da experiência e habilidades dos funcionários;
    \item Iniciativa: Os funcionários devem ter liberdade para propor ideias e tomar ações para melhorar o trabalho;
    \item Espírito de equipe: O trabalho em equipe e a cooperação são fundamentais para o bom funcionamento da organização.
\end{enumerate}

\noindent \textbf{Questão 7) O que foi o Movimento das Relações Humanas abordado por Elton Mayo?}\par
O Movimento das Relações Humanas, abordado por Elton Mayo, foi uma abordagem inovadora na administração que surgiu na década de 1930 como uma crítica à abordagem tradicional da Administração Científica. Esses experimentos tinham como objetivo investigar os efeitos das condições de trabalho na produtividade dos funcionários. Inicialmente, a pesquisa focava em aspectos físicos, como iluminação, temperatura e pausas, mas logo Mayo percebeu que outros fatores, como o ambiente social e as relações interpessoais, desempenhavam um papel significativo.\par

\noindent \textbf{Questão 8) Qual foi o foco/lema central da Teoria Comportamental de Simon?}\par
O foco da Teoria Comportamental de Simon era compreender o comportamento humano na tomada de decisões, reconhecendo as limitações cognitivas e buscando entender como as pessoas tomam decisões satisfatórias dentro das restrições e influências do ambiente organizacional.\par

\noindent \textbf{Questão 9) Explique a Teoria X e Y abordada por Simon.}\par
Nesta teoria, as atitudes e as ações dos gerentes em relação aos seus subordinados são influenciadas por essas percepções subjacentes sobre a natureza humana. Ele defendeu que a adoção da Teoria Y e a promoção de uma abordagem mais participativa e voltada para as pessoas resultam em maior satisfação no trabalho, maior motivação e melhores resultados organizacionais. Portanto, a Teoria X e Y representa duas visões contrastantes sobre a natureza humana e fornece um modelo conceitual para entender diferentes estilos de gestão e liderança.\par

\noindent \textbf{Questão 10) Explique a hierarquia das necessidades de Maslow abordada por Simon.}\par
A hierarquia das necessidades de Maslow é uma teoria que descreve uma série de necessidades humanas básicas, organizadas em uma hierarquia de cinco níveis (\emph{Necessidades fisiológicas, Necessidades de segurança, Necessidades sociais, Necessidades de estima, Necessidades de autorrealização}). Essa teoria sugere que, à medida que as necessidades em um nível inferior são satisfeitas, as necessidades no próximo nível se tornam mais proeminentes e motivadoras.

\noindent \textbf{Questão 11) Explique os aspectos da teoria comportamental em relação liderança. Comente seus estilos.}\par
A teoria comportamental da liderança, também conhecida como abordagem dos estilos de liderança, é uma perspectiva que se concentra no comportamento dos líderes em vez de traços inatos. Essa teoria sugere que o comportamento do líder desempenha um papel importante na eficácia da liderança e no desempenho dos seguidores.

\end{document}