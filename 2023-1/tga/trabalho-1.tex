% 1o trabalho de TGA
% Anderson Fraga - Marco 2023

\documentclass{article}
\usepackage[a4paper,includeheadfoot,margin=2.54cm]{geometry}
% \usepackage{cmbright}
\usepackage[OT1]{fontenc}
% \usepackage{antiqua}
% \usepackage[T1]{fontenc}
% \usepackage{notomath}
\usepackage{parskip}

\begin{document}

\underline{\textbf{Primeiro exercício de Teoria Geral de Administração}}\par
\textbf{Sistemas de Informação}\\
\textbf{Instituto Federal do Espírito Santo}\\
Campus Serra - Espírito Santo\par
\textbf{Teoria Geral da Administração}\\
Prof. Ronaldo Marques\par
Anderson A. Fraga (20222BSI0482)\\
\texttt{aafrg@tuta.io}\\  %\texttt formats the text to a typewriter style font


\noindent \textbf{Questão 1) Conceitue corretamente e explique sua importância para os negócios}\par
É a condução racional das atividades de uma organização seja ela lucrativa ou não. A Administração é imprescindível por tratar do planejamento, da organização, da direção e do controle de todas as atividades em uma divisão de trabalho que ocorra dentro de uma organização.

\noindent \textbf{Questão 2) Quais são as 4 etapas do Processo Administrativo? Explique cada uma delas incluindo o \emph{feedback}.}\par
\begin{itemize}
    \item Planejamento - definição de objetivos e recursos;
    \item Organização - disposição dos recursos em uma estrutura organizacional;
    \item Execução - realização dos planos; e
    \item Controle - verificação dos resultados.
\end{itemize}

\noindent \textbf{Questão 3) Explique as diferenças entre \emph{Eficiência, Eficácia e Efetividade}.}\par
\begin{itemize}
    \item Eficiência - relacionado ao uso dos meios e recursos;
    \item Eficácia - relacionado ao objetivo e resultados esperados;
    \item Efetividade - é atigir os objetivos utilizando a menor quantidade de recursos possíveis (\emph{eficiente e eficaz ao mesmo tempo}).
\end{itemize}
%\pagebreak
\noindent \textbf{Questão 4) A formação do \emph{Conhecimento Administrativo} é produzida pela observação e análise crítica da experiência prática das organizações e de seus administradores. Cite e explique \emph{as duas fontes de conhecimento e os três métodos} mais comuns para implementá-los.}
\par Em Administração, duas fontes de conhecimento que podem ser citadas são:
\begin{enumerate}
    \item \emph{Experiência prática}: o administrador dispõe de um acervo teórico que faz parte da cultura coletiva e é produto de transmissão de conhecimentos empíricos (\emph{adquiridos com a experiência}) desde que foi criada a primeira organização; e
    \item \emph{Métodos científicos}: aplicação da ciência à observação das organizações e dos administradores. Para tal, são conhecidos três métodos:
          \begin{enumerate}
              \item \textbf{\emph{Experimento}}: consiste em medir as consequências de uma alteração produzida em uma situação.
              \item \textbf{\emph{Levantamento Simples}}: levantamento de dados acerca do objeto de estudo. Utilizam-se, como exemplo de aplicação do método, questionários, entrevistas, observação direta.
              \item \textbf{\emph{Levantamento Correlacional}}: relação de causa e efeito entre determinados processos ou variáveis.
          \end{enumerate}
\end{enumerate}
\pagebreak
\noindent \textbf{Questão 5) O mercado para os artesãos após a Primeira Revolução Industrial mudou completamente. Em meados do século 18, o mundo deixa a manufatura e adota a maquinofatura quando as máquinas vão multiplicar o rendimento do trabalho e da produção. Surgem duas classes sociais: o empresário e o trabalhador ou proletariado.
    \underline{Explique os termos}: \emph{manufatura, maquinofatura, empresário e trabalhador/proletariado}.}
\par Podemos dividir os termos em \underline{classes sociais e modos de produção}:
\begin{itemize}
    \item \textbf{\emph{Classes sociais}}: o empresário - proprietário do capital (\emph{máquinas, matérias-primas, bens produzidos}) e o trabalhador ou proletariado - força de trabalho.
    \item \textbf{\emph{Modos de produção}}: a manufatura - produção realizada essencialmente pelas mãos da força de trabalho; a maquinofatura - após a Revolução Industrial, houve a mecanização da produção anteriormente feita por mãos dos trabalhadores (manufatura).
\end{itemize}

\noindent \textbf{Questão 6) Se a Administração sempre existiu de alguma forma (ver linha do tempo da Administração), explique como os \emph{filósofos, igreja católica / organização militar, a Revolução Industrial, os pioneiros/ empreendedores} deixaram suas contribuições que favoreceram a Administração.}
\begin{enumerate}
    \item A \emph{influência da Igreja Católica} pode ser descrita em:
          \begin{enumerate}
              \item Hierarquia escalar;
              \item Hierarquia de autoridade - um estado maior (\emph{assessoria}) e a coordenação funcional para assegurar integração.
          \end{enumerate}
    \item A \emph{influência da organização militar} pode ser descrita em:
          \begin{enumerate}
              \item Os fundamentos militares exerceram influência no desenvolvimento das teorias da Administração;
              \item Princípio da unidade de comando - \emph{cada subordinado só pode ter um superior};
              \item Escala hierárquica - os escalões hierárquicos de comando com graus de autoridade e responsabilidade;
          \end{enumerate}
\end{enumerate}

\noindent \textbf{Questão 7) Explique o processo proposto por Renè Descartes na sua obra \emph{O Discurso} e que é fundamental até hoje na Administração das empresas.}
\par Simplificadamente, são os passos ou preceitos:
\begin{enumerate}
    \item \textbf{\emph{Examinar as informações}}, verificando sua coerência e sua justificação.
    \item \textbf{\emph{Verificar a verdade}}, a boa procedência daquilo que se investiga - aceitar apenas o que seja verificável e confiável;
    \item \textbf{\emph{Análise}}, ou divisão do assunto em tantas partes quanto possível e necessário;
    \item \textbf{\emph{Síntese}}, ou elaboração progressiva de conclusões abrangentes e ordenadas a partir de objetos mais simples, progredindo até dados mais complexos;
    \item \textbf{\emph{Enumerar e revisar minuciosamente as conclusões}}, garantindo que nada seja omitido e que a conexão geral exista.
\end{enumerate}
\pagebreak
\noindent \textbf{Questão 8) A organização e a empresa moderna nasceram com a Revolução Industrial graças a alguns eventos. Quais são eles?}
\begin{enumerate}
    \item À ruptura das estruturas corporativas da Idade Média;
    \item Ao avanço tecnológico e sua aplicação à produção;
    \item A nova tecnologia dos processos de produção; e
    \item A prática e implementação de novas ideias.
\end{enumerate}

\noindent \textbf{Questão 9) Conceitue \emph{Indústria 4.0} e apresente alguns \emph{princípios para desenvolvimento e sua implementação} visando tornar os \emph{sistema de produção inteligentes}.}
\par É um conceito de indústria que engloba as principais inovações tecnológicas dos campos de automação, controle e tecnologia da informação, aplicadas aos processos de manufatura. Ao interligar máquinas, sistemas e ativos, são criadas redes inteligentes ao longo de toda a cadeia de valor que controla os módulos da produção, e que de forma autônoma aceleram as mudanças dentro do setor industrial.

\noindent \textbf{Questão 10) Explique as diferenças entre \emph{habilidades} e \emph{competências}.}
\begin{itemize}
    \item \textbf{\emph{Habilidades}} são aptidões que uma pessoa desenvolve para executar determinado papel ou função;
    \item \textbf{\emph{Competência}} consiste na junção e coordenação das habilidades com conhecimentos e atitudes.
\end{itemize}

\noindent \textbf{Questão 11) Explique o conceito de \emph{Estratégia e Gestão estratégica}.}
\begin{itemize}
    \item \textbf{\emph{Estratégia}} é a análise da situação presente e a sua mudança, se necessário, bem como a definição dos recursos atuais e dos necessários para uma finalidade/objetivo específicos.
    \item \textbf{\emph{Gestão estratégica}} é uma maneira de gestão que otimiza os processos de uma empresa buscando eficácia, eficiência e efetividade, realizando o chamado “\emph{planejamento estratégico}”.
\end{itemize}

\noindent \textbf{Questão 12) A TGA estuda a administração das organizações sob o ponto de vista de seis variáveis interdependentes. Quais são elas?}
\par A Teoria Geral da Administração estuda a administração das organizações sob o ponto de vista de seis variáveis interdependentes, das quais podemos citar:
\begin{enumerate}
    \item Pessoas;
    \item Ambiente;
    \item Tarefas;
    \item Estrutura;
    \item Tecnologia; e
    \item Competitividade.
\end{enumerate}

\noindent \textbf{Questão 13) No planejamento estratégico, as empresas precisam definir sua \emph{missão, princípios e valores} e também a \emph{sua visão}. Explique cada um deles.}
\begin{itemize}
    \item \textbf{\emph{Missão}}: é a razão de ser da empresa, o propósito pelo qual ela existe. A missão é uma declaração clara e concisa que descreve qual é o negócio da empresa, quem são seus clientes, quais são seus produtos ou serviços e qual é o valor que ela agrega à sociedade.;
    \item \textbf{\emph{Princípios e Valores}}: são as crenças e as convicções que guiam o comportamento da empresa. Eles refletem a cultura da organização e a forma como ela se relaciona com seus \emph{stakeholders}.
    \item \textbf{\emph{Visão}}: é a imagem do futuro que a empresa deseja construir. Ela deve ser \emph{clara, motivadora e inspiradora}. A visão deve descrever o estado futuro desejado para a empresa, indicando onde a organização quer chegar em um horizonte de tempo determinado.
\end{itemize}

\noindent \textbf{Questão 14) As empresas possuem diversos tipos de recursos. Explique o que vem a ser \emph{recursos tangíveis, recursos intangíveis e reputação}.}
\begin{itemize}
    \item \textbf{\emph{Recursos tangíveis}}: são aqueles que podem ser alcançados, tateados e/ou vistos, como materiais de produção, imóveis, dinheiro em caixa e estoques. Esses recursos são facilmente mensuráveis e podem ser administrados por meio de ferramentas de controle de estoque e fluxo de caixa.
    \item \textbf{\emph{Recursos intangíveis}}: são aqueles que não tateáveis (\emph{pertencem ao campo das ideias}), mas têm grande valor para a empresa, como a marca, a patente, a tecnologia, a cultura organizacional e os processos de produção. Esses recursos são mais difíceis de serem mensurados, mas são fundamentais para a diferenciação da empresa no mercado e para a sua competitividade.
    \item \textbf{\emph{Reputação}}: trata-se da relação de confiança entre a empresa e o consumidor. Ela é formada a partir da experiência dos clientes com os produtos ou serviços da empresa, da imagem que os colaboradores e fornecedores têm dela, da forma como a empresa se relaciona com a sociedade e do seu comportamento ético e socialmente responsável. A reputação é um ativo intangível valioso para a empresa, pois pode influenciar as decisões de compra dos clientes.
\end{itemize}

\noindent \textbf{Questão 15) Vários fatores deverão provocar profundos impactos sobre as organizações nos próximos anos. Assim, qual deve ser o papel essencial e fundamental da Administração contemporânea?}
\par A essência fundamental da Administração Contemporânea é a visão estratégica de cada operação ou atividade de forma sistêmica. \emph{Em outras palavras}: é imprescindível vislumbrar cada tarefa e cada atividade em um contexto ambiental mais amplo e que se modifica constantemente.


\end{document}